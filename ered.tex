\documentclass[]{article}
\usepackage[utf8]{inputenc}
\usepackage{graphicx}
\usepackage{amsmath}
\usepackage{amssymb}
\usepackage{epstopdf}

\title{Hold pályája}
\author{Furuglyás Kristóf}

\begin{document}

\maketitle
\newpage

\section{Föld-Hold rendszer}

Ptolemaiosz idejében idejében még elfogadott volt, hogy a Föld volt a világmindenség közepe. Manapság már tudjuk, hogy tévedett, de vannak követői így is\footnote{https://theflatearthsociety.org}. \\
Mindennek ellenére, ha felnézünk az éjjeli égboltra, és elgondolkodunk azon, hogy a Hold hogy kering, akkor elég csak a Föld gravitációs terét figyelembe venni.  

\section{Feladat}

A programozási feladatunk az volt, hogy a Föld-Hold rendszert kellett ábrázolni egy geocentrikus, síkbeli koordinátarendszerben. A programban explicit Euler módszerrel integráljuk ki az alábbi differenciálegyenletet:\\
$$
\underline{\ddot{r}}=\frac{M \gamma }{|\underline{r}|^3} \underline{r}
$$

ahol $\underline{r}$ a helyvektor, $\gamma$ a gravitációs állandó, $M$ pedig a Föld tömege.\\
A programkódot mellékeltem.

\paragraph{Kezdeti feltételek} Mivel a diffegyenletünk másodrendű, illetve két koordinátával dolgozunk, 4 kezdeti feltételt kellett megadni. A megoldásban a Hold perigeumból indul, megadott sebességgel (azaz y=0, x=perigeum, illetve y irányú sebessége a perigeumbéli sebessége, x irányú sebessége pedig 0).

\section{Megoldás}

\begin{figure}
\centering
\includegraphics[width=\textwidth]{ered.png}
\caption{A Föld-Hold rendszer explicit Euler módszerrel megoldva. Látható, hogy a megoldás nem jó, hiszen a Hold jelentős mértékben távolodik, mindössze néhány periódus megtétele után}
\label{fig:plot}
\end{figure}
A megoldást a \ref{fig:plot}. ábra mutatja. Jobb megoldás érdekében negyedrendű Runge-Kutta módszert érdemes használni.




\end{document}
